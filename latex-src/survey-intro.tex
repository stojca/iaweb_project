
%----------------------------------------------------------------
%
%  File    :  survey-recc.tex
%
%  Author  :  Mirza Kabiljagic, Stefan Rajinovic, Aleksandar Stojicic,
%             Inti Gabriel Mendoza Estrada
%
%  Created :  27 May 2019
%
%  Changed :  05 Dec 2019
%
%----------------------------------------------------------------

\chapter{Introduction}

\label{chap:Intro}

Going back, at the starting point of web development, its creators and
designers have not had the need to think about the look and design of
the page, because back then, their users and clients would mostly
access the site via the same device, that being a personal desktop
computer which would mostly share the same resolution and display
size. That means that they mostly needed to design and work around the
same concept. Also, they didn’t need to think about things like what
if an user entered the site with this device or this, so they could
basically use static and fixed positioning of elements.

However, todays' technologies, be it the mobile industry or technical
industry in general, the devices have come a long way, and are being
worked on really fast. What does that mean is that the programmers and
designers need to consider that people will have different ways of
using and browsing their pages, which require different approaches to
design their web pages. Nowadays, the clients and users have a broad
list of devices, which they can use to access a web page, be it a
tablet, mobile phone, laptop and even TVs \parencite{ITU}.

Basically, the most important thing is to make the web pages' user
interface as friendly as possible, so that the user does not lose much
time navigating around. And with that being said, web programmers need
to automatically think about making the web pages in that way, while
they are being accessed from those different devices mentioned
\parencite{SharkiFisher}.

Responsive Web Design is an approach, or more rather a technique, which
is used to construct a self-adapting web page, meaning that it adapts,
resizes, shrink and changes its' form in any way needed to perfectly fit
the web-accessing device used to visit it \parencite{HimanshuNegi}.

  



\section{How to achieve Responsive Web Design}

If a programmer wants his web site to be responsive, he has to use
following techniques:

\begin{itemize}
    \item Media queries
    \item Flexible images and media
    \item Fluid grid layouts
\end{itemize}


\section{Media queries}

Media queries are very important because with them it is possible to
specify with what device the web page is being accessed. The query
usually is made of a media type, and some conditions which can check if
the media type has some special property \parencite{A.A.Mohamed}.



\begin{lstlisting}[
    language=CSS, 
    float=tp,
    xleftmargin=0cm,              % no extra margins for floats
    xrightmargin=0cm,             % no extra margins for floats
    language=biblatex,
    basicstyle=\footnotesize\ttfamily,
    frame=shadowbox,
    numbers=left,
    label=list:MediaQueries,
    caption={[Media Queries]
    },
]
% An example of a media query, where one can say that for a screen of size 600px or less, the 
% background color is olive:

@media screen and (max-width: 600px) {
  body   {
    background-color: olive;
  }
}
\end{lstlisting}



Basically it allows defining specific CSS rules, depending on the
device used. From the example in Listing~\ref{list:MediaQueries}, with the
help of the parameter \propname{screen} the media type was specified
and using \propname{max-width} the condition was determined which
affects the device of the user. Now, there are many different
conditions which can be used, and they are really helpful, because
with them it is possible to adjust display settings for every device
there is.

For example, if the user is entering the website with his phone, it can
be automatically said that an image should have a lower resolution, thus
adjusting it better for his screen and saving some bandwith.

Another example would be using \propname{display:none} with which some
content can completely be hidden that should not be displayed.

\begin{lstlisting}[%
    language=CSS, 
    float=tp,
    xleftmargin=0cm,              % no extra margins for floats
    xrightmargin=0cm,             % no extra margins for floats
    language=biblatex,
    basicstyle=\footnotesize\ttfamily,
    frame=shadowbox,
    numbers=left,
    stringstyle=\color{blue},
    label=list:Media2,
    caption={[Media Queries 2]
    },]
% An example of using display: none:
<!DOCTYPE html>
<html>
  <head>
    <style>
      h1.none {
        display: none;
      }
    </style>
  </head>

  <body>
    <h1 class="none">This will not be displayed</h1>
  </body>
</html>
\end{lstlisting}



If this Listing~\ref{list:Media2} was ran in the browser, the
header would not be seen, because it was decided to not be shown. If
\propname{display:none} was changed to \propname{display:block}, our
header would be visible.

Include a list of \propname{all} the relevant papers and resources you
have found and mark those you have chosen to focus on. Make sure
\propname{all} the papers and resources you found or were given appear in
the bibliography.




\section{Fluid Grid Layouts}

Normally, websites are constructed in a layout which is grid-built,
because they are easier to handle in different kind of devices. Also
what is specific for that kind of layout is that it is built using divs,
HTML tables and so on. Now, on top of that responsive web design comes
in play\parencite{B.Frain}. With it, it is possible to use
percentage-based element sizing which is easier than using pixels.

According by (Abdulrehman,2015) a flexible grid-layout is one of the
foundations of responsive design. Using the term "grid" does not mean
necessarily that a grid framework needs to be used. What it means is
that specific CSS tricks are used for positioning. Also, he suggest that
using pixels for the measurement unit should be stopped, because pixels
can vary.

For example, on one device they can be one point or dot, on another
device it can be a few more, and therefore unreliable. So, what needs to
be done is that, if pixels are used, they should be converted using a
formula stated by (Marcotte, 2010) in  Dan Cederholm's book named
"Handcrafted CSS":

\begin{ceqn}
    \begin{align}
        Target \div Context = Result
    \end{align}
\end{ceqn}

It is asserted by (Pettit, 2012) that, to use this formula, the context
of the element needs to be divided by the target element.



\section{Flexible images and media}

With this approach, any type of media,images depending on the screen
resolution will resize,collapse,crop and adjust accordingly to it, and
to achieve this it is possible to use the same thing mentioned in the
previous section, relative units, if the screen becomes too small then
hide the image completely or by cropping some parts of the image, again
in the case if the screen becomes too small\parencite{HTML5Tutorial}.

\subsection{Relative measurements}

Instead of using pixels,relative dimensions can be used, percentage
based. So instead of specifying the view and dimensions in pixels, one
can specify for example 60\%, which would result in resizing or
reshrinking of the image based on the resolution of the device
accordingly to fill 60\% of the page.

\begin{figure}[tp]
    \centering
    \includegraphics[keepaspectratio,width=\linewidth,height=\halfh]
    {images/relative.png}

    \caption[Relative Image Dimensions]{
        Relative Image Dimensions.
        \imgcredit{ Image taken from
          \href{https://www.researchgate.net/publication/308182637_Responsive_Web_Design_Techniques}{ [research paper]}
        , author was asked for permission.}
    }
    \label{fig:relative}
\end{figure}



\subsection{Cropping}

Another option is cropping, which crops the picture to a certain width
with the help of CSS.

\begin{figure}[tp]
    \centering
    {%
        \includegraphics[width=0.65\linewidth]
        {images/alig1.png}%
        \label{alig1}%
        \caption{Picture with original size}
    }
    \hfill
    {%
        \includegraphics[width=0.3\linewidth]
        {images/alig1.png}%
        \label{alig2}%
        \caption{Picture when cropped}
    }

    \caption[Image Cropping]
    {
        Image cropping.
        \imgcredit{Photo taken by author.}
    }
    \label{img_cropping}
\end{figure}

In Figure \ref{img_cropping}, we see the result of
tweaking the image a bit with just a few CSS commands. It enables us 
to crop the picture. If someone is browsing the website with a
smaller device, the image does not have to be hidden, just its size is
changed.

\begin{lstlisting}[%
    float=tp,
    language = CSS, 
    xleftmargin=0cm,              % no extra margins for floats
    xrightmargin=0cm,             % no extra margins for floats
    language=biblatex,
    basicstyle=\footnotesize\ttfamily,
    frame=shadowbox,
    numbers=left,
    label=list:ImgCrop,
     stringstyle=\color{blue}
    ,
    caption={[Image Cropping]
    },
]
% An example of a media query, where one can say that for a screen 
% of size 850px or less, the background is black:
figure{
  width:18.75rem; /*container-width*/
  overflow:hidden; /*hide bounds of image */
  margin:0;   /*reset margin of figure tag*/
}

figure img{
  display:block; /*remove inline-block spaces*/
  width:100%; /*make image streatch*/
}
\end{lstlisting}
